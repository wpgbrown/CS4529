\chapter{Requirements\label{chap:requirements}}
\settocdepth{section}

The following requirements chapter is split into requirements for the data collection, implementation of the automatic reviewer recommender tool and finally requirements that don't fit into the previous two categories. If parts of the requirements are not fulfilled, it does not necessarily indicate failure. A reflection on the achievement of these requirements is included in Chapter~\ref{chap:conclusion}.

\section{Data Collection Requirements}
\subsection{Functional}
I have the following functional requirements for the data collection:
\begin{itemize}
    \item The data collected should cover all repositories that start with the `mediawiki' prefix on MediaWiki Gerrit.
    \item The data should be saved using JSON where possible.
    \item The data collected should be in one go where possible
\end{itemize}

The need for data collection over all repositories that start with the  `mediawiki' prefix is because the data collected should allow the system to perform recommendations relatively quickly and without the need to perform HTTP requests when training and evaluating. If there was a need to perform HTTP requests to the REST APIs it could easily become an abuse of the resources they provide and could potentially lead to access being removed.

Downloading all the data in one go but appropriately rate limited is better for the user who wants to get a recommendation, better for the maintainer who wants to potentially re-train multiple times without needing to make additional HTTP requests that add run time and for the service who only is asked for all the data once.

The use of JSON where possible is to allow manual inspection of the collected data using a text editor program. In some cases this isn't possible due to the data being complied down at run-time or needing to be pickled, but where possible saved data is placed in JSON files.

\subsection{Non-functional}
I have the following non-functional requirements for the data collection:
\begin{itemize}
    \item The data set and tools to collect it should be adaptable to use in future research surrounding MediaWiki.
    \item The types of data collected should be broad.
\end{itemize}

Making the data set and the tools adaptable for future research is important as the data collected here could be used in future work which does not directly relate to automatically recommending users as reviewers for a change but otherwise would need the data set collected by this project. With this requirement, the potential impact of the project is kept open to more than just automatically recommending reviewers.

\section{Recommender Requirements\label{section:recommender-requirements}}
The implementation of the proposed solutions should meet the following functional and non-functional requirements. The requirements inform our decision on whether a recommender implementation is complete for the purposes of this project.

\subsection{Functional}
I have the following functional requirements for each reviewer recommendation implementation:
\begin{itemize}
    \item The implementations must have a command line interface.
    \item The tool must be able to take pre-downloaded change information and make recommendations based on that without the need for an internet connection.
    \item The tool must be able to take unique IDs for a change and then download the change information for that change, and then make recommendations.
    \item The tool must produce a configurable number of reviewers ordered by their strength of recommendation.
    \item After appropriate adjustment, training and testing the implementation must be able to produce recommendations without the need for human interaction
\end{itemize}

The need for a command line interface means that the tool can be run by other programs without the need to simulate user inputs. This would also make it easier to test the system as a ``opaque box'' where the internals of the system cannot be accessed except for only the inputs and outputs. By allowing recommendations to be performed using pre-downloaded change info, the system can make recommendations without needing to make HTTP requests. Therefore, testing can be done offline and run much faster without the need to wait for a resource.

The requirement to produce recommendations without the need for human interaction is so that the tool can after being set up produce recommendations on demand when changes are submitted for review.

\subsection{Non-functional}
I have the following non-functional requirements for each reviewer recommendation implementation:
\begin{itemize}
    \item The tool should provide good-quality recommendations for reviewers to be added for a given change.
    \item The tool should work over a wide range of repositories in the MediaWiki project.
    \item The core tool should be built with adaptability to other projects in mind where possible.
    \item The tool should avoid duplication of users in the results.
\end{itemize}

The first requirement is crucial to the success of the project. Without this, the tool will likely not provide benefits over the existing implementations as discussed in Chapter~\ref{chap:background-and-related-work}. The requirement of making the tool work over a wider range of repositories helps differentiate this from previous research. Duplication of results should be avoided where possible to help make predictions more accurate.

\section{Project Level Requirements}

I have a number of requirements for the project as a whole:
\begin{itemize}
    \item The project should produce an implementation that makes use of machine learning.
    \item The project should answer the research question while leaving space for future research.
\end{itemize}

An implementation of a machine learning method helps make this project stand out from previous research. The second point of answering the research question is important as the project should aim to finish what it originally intended to do.

\settocdepth{subsection}
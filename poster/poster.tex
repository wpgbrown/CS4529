\documentclass[a0,portrait]{lab-poster}

\usepackage[british]{babel}    % Language Configuration

% \newcommand\itemadjust{\itemsep.5em \parskip0pt \parsep0pt}

\title{Your Title Here}
\author{John Student}
% \major{CS4525: Joint Honours Computer Project}
\major{CS4529: Single Honours Computing Project}
% \major{CS4594: Joint Honours Computing-Physics Project}
\supervisor{Mary Supervisor}

\themecolor{NavyBlue}
\unilogo{images/logoAberdeen.pdf}
\lablogo{images/logoAberdeen.pdf}

\begin{document}
\maketitle

%---------------------------------------------------------------

\begin{multicols}{2} 
%---------------------------------------------------------------
%	Motivation
%---------------------------------------------------------------
% \color{NavyBlue}
\section*{Section 1}
% \color{Black}
\Large
\justifying

\begin{itemize}
	\item Point 1;
	\item Point 2;
	\item 
\end{itemize}

%---------------------------------------------------------------
%   AGENTES
%---------------------------------------------------------------
\section*{\huge Section 2}

\begin{itemize}
	\item Idea 1; and
	\item Idea 2.
\end{itemize}

%---------------------------------------------------------------
%	Algorithm / Technique
%---------------------------------------------------------------
\section*{\huge Section 3}

\begin{itemize}
	\item Some interesting description of your point one;
	\item Programmes developed using language XYZ as shown in Listing~\ref{alg:example-hello-world}, define set of beliefs, plans, triggering events, and actions that the agent runs in the environment;
	\item Definitions from Table~\ref{tab:syntax-agentspeak} specify these sets.
\end{itemize}

\vspace{13mm}


\begin{itemize}
	\item More Ideas...:
	\begin{enumerate}
		% [leftmargin=2em]\itemadjust
		\item Idea 1;
		\item Idea 2; 
		\item Idea 3; and
		\item Idea 4.
	\end{enumerate}	
	\item Figure~\ref{fig:myfig} illustrates this.
\end{itemize}
\vspace{13mm}

\begin{center}
	%\includegraphics[width=0.99\linewidth]{fig/myfig.pdf}
	\Huge Figure Here (include commented out)
	\captionof{figure}{Figure example.}
	\label{fig:myfig}
\end{center}	


\section*{Section 3}

\begin{itemize}
	\item More Ideas...:
	\begin{enumerate}
		% [leftmargin=2em]\itemadjust
		\item Idea 1;
		\item Idea 2; 
		\item Idea 3; e
		\item Idea 4.
	\end{enumerate}	
	\item Figure~\ref{fig:myfig} illustrates this.
\end{itemize}


\vspace{13mm}
\noindent\begin{minipage}{.235\textwidth}
	\begin{minipage}{\textwidth}
		\lstset{style=codeStyle}
		\begin{lstlisting}[language=Prolog, label={alg:example-hello-world}, caption={AgentSpeak(L) programme example.}]
		/* Agent helloWorld */
		/* Initial beliefs and rules */
		
		/* Initial goals */			
		!start.
		/* Plans */
		+!start : true <- aloha; ?continue(true);      !run (agentspeak).
		+!run(A) : true <- mahalo(A).
		\end{lstlisting}
	\end{minipage}\hfill
	\vspace{7mm}
	
	\begin{minipage}{\textwidth}
		\lstset{style=codeStyle}
		\begin{lstlisting}[language=Prolog, label={alg:exemplo-projeto-hello-world}, caption={AgentSpeak(Py) programme example.}]
		/* Project Name */
		helloWorld:
		// List of agents
		agents = [helloWorld]
		environment = HelloWorldEnv
		\end{lstlisting}
	\end{minipage}\hfill
\end{minipage}\hfill
\begin{minipage}{.235\textwidth}
	\lstset{style=codeStyle}
	\begin{lstlisting}[language=Python, label={alg:exemplo-environment}, caption={Environment description in Python.}]
	from environment import *
	lt_continue = parse_literal('continue(true)')
	
	class HelloWorldEnv(Environment):
		def __init__(self):
			Environment.__init__(self)
		
		def execute_action(self, agent_name, action):
			self.clear_perceptions()
			getattr(self, action.functor)(list(action.args))
		
		def aloha(self, *args):
			self.add_percept(lt_continue)
			print('Aloha HelloWorldEnv!')
		
		def mahalo(self, *args):
			print('Mahaloing with %s!' % ", ".join(map(str, *args)))
	\end{lstlisting}
\end{minipage}

\section*{\huge Section 4}
      
Current font sizes:
\begin{itemize}
	\item {\tiny \verb|\tiny| Font Size: \showfontsize}
	\item {\scriptsize \verb|\scriptsize| Font Size: \showfontsize}
	\item {\footnotesize \verb|\footnotesize| Font Size: \showfontsize}
	\item {\small \verb|\small| Font Size: \showfontsize}
	\item {\normalsize \verb|\normalsize| Font Size: \showfontsize}
	\item {\large \verb|\large| Font Size: \showfontsize}
	\item {\Large \verb|\Large| Font Size: \showfontsize}
	\item {\LARGE \verb|\LARGE| Font Size: \showfontsize}
	\item {\huge \verb|\huge| Font Size: \showfontsize}
	\item {\Huge \verb|\Huge| Font Size: \showfontsize}
	\item {\veryHuge \verb|\veryHuge| Font Size: \showfontsize}
\end{itemize}


%---------------------------------------------------------------
%	REFERENCES
%---------------------------------------------------------------
% Uncomment below if you want to use bib references in your poster
%\vspace{-10mm}
%\large
%\color{NavyBlue}
%\color{Black}
%\raggedright
%\bibliographystyle{plain}
%\bibliography{poster}

\end{multicols}

%----------------------------------------------------------------------------------------
\end{document}